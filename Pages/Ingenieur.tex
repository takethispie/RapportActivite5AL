\chapter{Le Métier d'ingénieur informatique}
L'ingénieur informatique est un professionnel appliquant les principes de l'ingénieurie au développement de logiciels,
il possède de nombreuses compétences n'appartenant pas au seul domaine de la programmation tel que les bonnes pratiques 
(extreme programming, KISS, SOLID), connaissances approfondies de l'algorithmie, des structures de données 
et de la scalablilité, le tout associé à une veille technologique regulière pour toujours savoir
repondre au besoin de manière pertinente.

\section{Apport de mon parcours}
\subsection{Alternance}
L'alternance pendant ma 4ième année m'a apporté beaucoup de compétence en développement avec l'apprentissage 
du framework angular, ionic et asp.net 
j'ai forgé des compétences en web qui était quasiment inexistante auparavant car je n'avais fait que 
du développement d'application de bureau pendant mes stages en BTS, comprendre plusieurs paradigme 
et plusieurs types de stack technologiques (web, mobile, desktopp) est très important pour ingénieur
qui doit avoir une multitude d'outils technologiques à sa ceinture \newline

Lors de la 5ième année je suis rapidement arrivé au bout de ce que pouvais m'offrir l'entreprise en  
terme d'acquisition de compétences, en effet l'entreprise ou j'ai fait mon alternance 
n'a pas une taille permettant de recruter une équipe d'ingénieurs et dans une équipe 
de 2 personnes difficile d'apprendre l'importance de communiquer ou les compétences en 
communication essentielles pour un ingénieur. \newline

Une autre problématique s'est posé lors de ma 5ième année qui fut à la fois un frein et bénefique pour 
mes compétences mais uniquement grâce à ma curiosité intellectuelle et mon envie de me surpasser là dans ce quoi 
je suis compétent, il s'agit de la qualité abyssale des projet sur lequels j'ai été amené à travailler 
en dehors de ma mission principale. En effet la qualité du code est quasi-nulle, aucun test unitaires, les principes 
considérés comme primordial de la programmation tel que SOLID ou KISS ne sont pas respectés, ni le paradigme 
même des langages utilisés. \newline

Travailler sur un code qui est de niveau, tout au plus, débutant associé à l'absence de toute 
gestion de projet ne permet pas d'acquérir des compétences via un acteur exterieur, 
l'effet insoupçonné est que j'ai appris a reconnaître avec beaucoup plus de rapidité 
et de pragmatisme les défaut de l'architecture d'un logiciel, de plus j'ai appris 
de façon très poussé la refactorisation de code et comment trouver les "code smells" pour 
les corriger. J'ai aussi fait une veille technologique journalière poussé sur l'architecture 
et les bonnes pratiques pour pouvoir l'appliquer sur ma mission principale et 
les projets où je pouvais faire des modifications retro-compatibles. 


\newpage

\subsection{Parcours Scolaire}
Mon parcours scolaire n'a pas été un parcours typique: j'ai fait la filière bac STI2D avant de faire un BTS SIO
puis une Licence PRISM pour finir par ces 2 années à l'ESGI, je pense que ce type de cursus m'a permis 
d'avoir un apprentissage très pratique, mes lacunes en mathématiques sont fortement compensé par de fortes compétences 
techniques. \newline 

Ces 2 dernières années à l'ESGI m'on appris une quantité importante de connaissances essentielles au métier 
d'ingénieur que je n'aurais pas eu en travaillant et apprenant tout seul, je pense notamment aux cours sur 
le machine learning qui m'ont permis de comprendre le fonctionnement du machine et deep learning, 
des techniques en forte croissance avec lesquelles je serais forcément amené à rentrer en contact. \newline 

les cours d'algorithmie avancés m'ont permis de me perfectionner là où j'avais de forte lacunes mais 
qui sont tout aussi importantes pour esperer surpasser le simple programmeur, 
les autres cours qui n'etaient pas autant pratiques m'ont permis entre autres d'étendre mes 
connaissances et mon champ de compréhension.  \newline 

toutes ces années m'ont permis d'acquérir un bagage technique mais surtout un niveau de compréhension 
qu'il n'est pas aussi facile d'atteindre en autodidacte, ce sont de plus des connaissances théoriques tout 
comme pratique, la philosophie d'education par projet de l'ESGI ayant eu un fort impact.

\subsection{Ce qui fait de moi un bon ingénieur}

Je pense que ce qui fait de moi un bon ingénieur est, en plus de mon bagage scolaire et professionnel qui me permet de
ne pas juste voir le code mais voir un problème auquel il faut trouver une solution et eriger une architecture 
qui répond au besoin mais aussi à des critère de qualité et de design permettant une maintenabilité et extensibilité 
optimale, 
une curiosité intellectuelle pour les nouvelles choses et nouvelles technologies
mais aussi ma passion pour le programmation depuis 12 ans et comme toute passion, elle pousse 
à devenir meilleur.

\section{Projet professionnel}
Mon projet professionnel est déjà lancé puisque j'ai déjà signé mon CDI chez SOAT une 
société de conseil dans laquelle je vais être ingénieur d'etudes, je souhaite évoluer 
au sein de cette entreprise et monter le plus rapidement possible en compétence technique 
mais aussi surtout dans le management et les méthodes agiles
puis eventuellement plus tard travailler dans un éditeur de logiciel / un client final.