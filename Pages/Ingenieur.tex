\chapter{Le Métier d'ingénieur informatique}
L'ingénieur informatique est un professionnel appliquant les principes de l'ingénieurie au développement de logiciels,
il possède de nombreuses compétences n'appartenant pas au seul domaine de la programmation tel que les bonnes pratiques 
(extreme programming, KISS, SOLID), connaissances approfondies de l'algorithmie, des structures de données 
et de la scalablilité, le tout associé à une veille technologique regulière pour toujours savoir
repondre au besoin de manière pertinente.

\section{Apport de mon parcours}
\subsection{Alternance}
L'alternance pendant ma 4ième année m'a apporté beaucoup de compétence en développement avec l'apprentissage 
du framework angular, ionic et asp.net 
j'ai forgé des compétences en web qui était quasiment inexistante auparavant car je n'avais fait que 
du développement d'application de bureau pendant mes stages en BTS, comprendre plusieurs paradigme 
et plusieurs types de stack technologiques (web, mobile, desktopp) est très important pour ingénieur
qui doit avoir une multitude d'outils technologiques à sa ceinture \newline

Lors de la 5ième année je suis rapidement arrivé au bout de ce que pouvais m'offrir l'entreprise en  
terme d'acquisition de compétences, en effet l'entreprise ou j'ai fait mon alternance 
n'a pas une taille permettant de recruter une équipe d'ingénieurs et dans une équipe 
de 2 personnes difficile d'apprendre l'importance de communiquer ou les compétences en 
communication essentielles pour un ingénieur. \newline

Une autre problématique s'est posé lors de ma 5ième année qui fut à la fois un frein et bénefique pour 
mes compétences mais uniquement grâce à ma curiosité intellectuelle et mon envie de me surpasser là dans ce quoi 
je suis compétent, il s'agit de la qualité abyssale des projet sur lequels j'ai été amené à travailler 
en dehors de ma mission principale. En effet la qualité du code est quasi-nulle, aucun test unitaires, les principes 
considérés comme primordial de la programmation tel que SOLID ou KISS ne sont pas respectés, ni le paradigme 
même des langages utilisés. \newline

Travailler sur un code qui est de niveau, tout au plus, débutant associé à l'absence de toute 
gestion de projet ne permet pas d'acquérir des compétences via un acteur exterieur, 
l'effet insoupçonné est que j'ai appris a reconnaître avec beaucoup plus de rapidité 
et de pragmatisme les défaut de l'architecture d'un logiciel, de plus j'ai appris 
de façon très poussé la refactorisation de code et comment trouver les "code smells" pour 
les corriger. J'ai aussi fait une veille technologique journalière poussé sur l'architecture 
et les bonnes pratiques pour pouvoir l'appliquer sur ma mission principale et 
les projets où je pouvais faire des modifications retro-compatibles. 


\newpage

\subsection{Parcours Scolaire}


\subsection{Ce qui fait de moi un bon ingénieur}

\section{Projet professionnel}
Mon projet professionnel est déjà lancé puisque j'ai déjà signé mon CDI chez SOAT une 
société de conseil dans laquelle je vais être ingénieur d'etudes, je souhaite évoluer 
au sein de cette entreprise et monter le plus rapidement possible en compétence technique 
mais aussi surtout dans le management et les méthodes agiles
puis eventuellement plus tard travailler dans un éditeur de logiciel / un client final.