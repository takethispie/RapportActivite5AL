\chapter{Mission Principale}
Ma mission lors de cette 5ième année a été de continuer le développement de l'application mobile Callibri Mobile
que j'ai commencé vers le milieu de la 4ième année, cette application est la refonte complète de l'ancienne 
application mobile qui n'est qu'un wrapper de navigateur autour du site web2 en responsive,
aujourd'hui les application mobiles étant un passage obligatoire pour fournir aux clients une experience 
correcte d'utilisation de nos services et pour rester compétitif face à nos concurrent 
il était nécéssaire de réaliser une application qui soit pensé pour une interface de smartphone et 
non juste une adaptation. \newline

lors de la 4ième année je me suis familiarisé avec Ionic et Angular, les deux framework que j'ai utilisé pour 
le développement de l'application, Ionic est un framework qui permet de créer des application mobile 
iOS et Android mais aussi desktop via electron en utilisant des technologies web.\newline

Au début du développement Ionic n'était qu'à ses balbutiements avec la version 1.3 et utilisait uniquement 
le framework angular, j'ai du très tôt dans le cycle de déveleppement porter la version vers ionic 2 qui n'etait pas retro 
compatible.\newline

Lors de ma 5ième année Ionic a subis de nombreux changements, le plus notable étant le changement de la version 3 
à la version 4, ce dernier est devenu framework-agnostic c'est à dire qu'il ne dépend plus d'un framework 
web sousdjacent spécifique mais peut être utilisé par les autres framework web les plus connus tel que 
React et vuejs mais aussi en standalone avec du javascript pure. \newline 

Comme nous avant passé plus d'un an à développer l'application en utilisant le framework Angular nous 
avons décider de continuer ce qui n'a pas empecher le fait que le changement de version 
a entrainé de nombreux changement qui fut le but principal de ma mission de 5ieme année. \newline

En passant à sa version 4, Ionic apporte grand nombre de changement:

\begin{itemize}
	\item passage d'angular 2+ à angular 7+: un grand bon en avant en terme de performance, ce qui 
	est très rechercher pour le développement hybride d'application mobile car cela permet de reduire 
	la différence de fluidité avec une application native. \newline

	\item utilisation du router angular en lieu et place du navcontroller de ionic:
	l'équipe de développement de Ionic à decider de permettre d'utiliser le meilleur 
	de chaquer framework du stack et le routeur angular en fait partie, cela fut une 
	partie assez difficile à mettre correctement en place car l'application 
	callibri mobile à une grande quantité de page et de modales. \newline 

	\item apparition du shadow-dom: un nouveau système utilisé dans le rendu du DOM,
	le shadow DOM est comme un subtree du DOM, il permet notammenet de simplifier 
	le fonctionnement des règles CSS mais surtout d'augmenter les performance de rendu 
	de l'html qui sont loin d'être optimales dans le cas d'un smartphone. \newline
	\begin{figure}[!h]
		\centering
		\includegraphics[width=1\linewidth]{Images/shadow}
		\caption{fonctionnement du shadow dom}
		\label{fig:archhexa}
	\end{figure}

	\item De nombreux changement dans le fonctionnement des composants Ionic et de leur 
	syntax ce qui a du nécéssiter la réecriture d'une bonne partie des fichiers html 
	de l'application plus un changement de l'architecture des fichiers du projet 
	pour refleter ces divers changement. \newline
\end{itemize}

Tout ces changments ont aussi été l'occasion de perfectionner le look de l'application 
et l'intuitivité de ses interfaces, le fastmenu qui etait un menu d'accès rapide sous 
forme de cercle disponible sur toutes les pages de l'application a été supprimé,
il prenait beaucoup de place inutilement pour être une duplication des contrôles 
déjà disponibles. \newline 

\section{Agenda}
Ce fut la partie de l'application qui a subis le plus de changement, anciennement l'agenda 
utilisait du code de très mauvaise qualité pour répondre à un besoin technique presque inexistant.
l'agenda utilise deux vues: la vue jour et la vue mois, dans l'ancienne version majeure
de l'application un système de cache complexe était utilisé pour charger une semaine entière 
et ainsi ne pas avoir à faire des appel regulier au serveur, pour l'affichage 
un système de slide tout aussi complexe était utilisé pour faire correspondre le jour affiché 
avec un des 7 jours en cache, cet ensemble était très mal écrit, tout à fait intestable 
et peu performant. \newline

pendant ma 5ième année j'ai refait cette page agenda, qui est la page la plus importante 
de tout l'application, en supprimant ce système de cache qui était plus source d'erreur 
et de difficulté de maintenabilité qu'autre chose puisqu'au final la requête pour charger 
un jour de l'agenda depuis le serveur est très simple et rapide. \newline

nous avons supprimer le système de slide par un système simple d'animation in \& out css 
qui donne l'impression que la page part sur le côté de l'écran pour laisser la page suivante
arriver du côté opposé, en plus de performance multiplié par un facteur 10, le code à été 
fortement réduit, l'animation est utilisé pour cacher une partie du chargement de la
nouvelle journée de l'agenda. \newline

\begin{figure}[!h]
	\centering
	\includegraphics[width=0.3\linewidth]{Images/agenda}
	\caption{les changements visuels de la page agenda son mineur et ne deroutent pas l'utilisateur }
	\label{fig:archhexa}
\end{figure}

\section{Contact}

la page de contact a subis de nombreux changements aussi mais plus du côté du code 
que l'interface graphique, le système de recherche de contact à été optimisé du côté 
du serveur pour ainsi avoir une latence de réponse plus faible et des résultats 
avec moins d'erreurs, l'affichage de la page de recherche est plus fluide car j'ai 
changé l'architecture des données pour utilisé de manière plus poussée le reactive 
programming et avoir l'état de l'interface graphique qui change correctement en 
fonction des données ce qui parfois n'était pas le cas avec l'ancienne version 
ou une mauvaise synchronisation entre par exemple le spinner de chargement 
et le chargement des données était differé de plusieurs centaines de millisecondes 
ce qui a un impact conséquent sur l'impression de fluidité. \newline

Le transfert d'information et de contact entre les différentes pages utilisant un contact tel
que la liste de contact, l'ajout de contact et la selection de contact depuis l'ajout 
d'un rendez-vous dans l'agenda à été réduite à son minimum pour simplifier le tracage 
des état de l'application: auparavant la liste de contact envoyait l'id du contact 	
à la page de details du contact qui chargeait les informations du contact et en cas de modification 
du contact renvoyait une commande à la page qui executait la commande qui consistait 
en un appel http pour modifier le contact puis le mettre à jour dans la liste de contacts.
\newline

J'ai refactorisé le code et notamment transformé de nombreuses pages en modales y compris la page
des détails d'un contact, dans la version actuelle de l'application la page liste de contact 
ouvre une modales avec l'id du contact et lors de modifications c'est la modale
elle même qui fait des appels http, la modale retourne un resultat vide lors de sa fermeture 
ou un objet contact si des modifications ont été faites pour pouvoir mettre à jour la liste, 
ce fonctionnement de type master-details est aussi présent pour les instructions, les messages 
et l'agenda et ce comportement y a été reproduit de la même manière dans ces dernières. 

\begin{figure}[!h]
	\centering
	\includegraphics[width=0.35\linewidth]{Images/contact}
	\caption{Page des contacts}
	\label{fig:archhexa}
\end{figure}



\newpage
\section{Bilan et recul sur la mission}
Pour cette mission j'ai eu totale liberté dans mes choix qui fut à la fois un avantage mais 
aussi une preuve du manque de gestion de projet au sein de l'entreprise, je n'avais 
aucun cahier des charges et les règles métiers n'etaient pas clairement définies,
j'ai du donc établir par moi même le cahier des charge et lister les nouvelles fonctionnalitées 
en me basant sur l'ancien site web et son code qui n'a d'ailleurs pas facilité la tache. \newline

j'ai du aussi me baser sur le website et m'auto-former aux bonnes pratiques 
à appliquer à l'expérience utilisateur et l'interface graphique puisqu'une application
n'a pas du tout les mêmes contraintes qu'un site web. \newline

Lors de mon rapport de 4ième année j'avais tiré la conclusion que javascript et son 
écosystème est assez médiocre en terme de qualité des outils et framework, 
un an plus tard mon avis a un peu changé, je pense qu'il est possible de trouver 
des framework de qualité, comme Ionic avec sa dernière version, Angular ou React 
et je pense finalement avoir fait les bon choix technologique pour cette application 
pour le long terme mais malgré cela je ne pense pas que javascript serait mon premier 
choix pour autre chose que du web.
 \newline




