\newglossaryentry{API}{
    name={API},
    description={Application Public Interface}
}

\newglossaryentry{Callibri}{
    name={Callibri},
    description={Logiciel de gestion d'agenda développé par Eurice}
}

\newglossaryentry{framework}{
    name={framework},
    description={Désigne un ensemble cohérent de composants logiciels structurels, qui sert à créer les fondations ainsi que les grandes lignes de tout ou d’une partie d'un logiciel}
}

\newglossaryentry{Angular}{
    name={Angular},
    description={Framework mobile crée par google, réecriture complète d'AngularJS}
}

\newglossaryentry{LINQ}{
    name={LINQ},
    description={Language INtegrated Query (Requête intégrée au langage) est un composant du 
    framework .NET de Microsoft qui ajoute des capacités d'interrogation sur des données aux langages .NET en utilisant une syntaxe proche de celle du SQL}
}

\newglossaryentry{front-end}{
    name={front-end},
    description={Le type d'application avec laquelle l'utilisateur intéragit directement}
}

\newglossaryentry{Node}{
    name={Node},
    description={NodeJS ou simplement Node est un framework permettant d'executé du code JavaScript côté serveur et non plus seulement du côté client 
    (dans le navigateur)}
}

\newglossaryentry{NPM}{
    name={NPM},
    description={Acronyme pour Node Package Manager, il s'agit du gestionnaire de paquet de Node}
}

\newglossaryentry{AES}{
    name={AES},
    description={Advanced Encryption Standard ou AES (soit « standard de chiffrement avancé » en français), 
    aussi connu sous le nom de Rijndael, est un algorithme de chiffrement symétrique}
}

\newglossaryentry{RSA}{
    name={RSA},
    description={Le chiffrement RSA (nommé par les initiales de ses trois inventeurs) est un algorithme de cryptographie asymétrique, 
    très utilisé dans le commerce électronique, et plus généralement pour échanger des données confidentielles sur Internet. }
}